\myprogram{{node\_deg\_over\_time.py}}
          {Time evolution of the degree of a node in a growing graph.}  
          {$<$layer$>$ $<$arrival\_times$>$ $<$node\_id$>$
          [$<$node\_id$>$ ...]}

\mydescription{Compute the degree $k_{i}(t)$ of node $i$ in a growing
          network as a function of time. The file \textit{layer}
          contains the edge list of the final network. Each line of
          the file is in the format:
  
  \hspace{0.5cm}\textit{src\_ID} \textit{dest\_ID}
  
  where \textit{src\_ID} and \textit{dest\_ID} are the IDs of the two
  endpoints of an edge.
  
  The file \textit{arrival\_times} is a list of node arrival times, in
 the format:

 \hspace{0.5cm} \textit{time\_i node\_i}
 
 where \textit{time\_i} is the time at which \textit{node\_i} arrived
 in the graph. Notice that \textit{time\_i} must be an integer in the
 range [0, N-1], where N is the total number of nodes in the final
 graph.

 The third parameter \textit{node\_id} is the ID of the node whose
 degree over time will be printed on output. If more than
 one \textit{node\_id} is provided, the degrees over time of all the
 corresponding nodes are printed on output.
 }


\myreturn{The program prints on \texttt{stdout} a list of lines in the
 format:

 \hspace{0.5cm} \textit{t kit}
 
 where \textit{kit} is the degree of node \textit{i} at
 time \textit{t}. The first line of output is in the format:

 \hspace{0.5cm} \textit{\#\#\#\# node\_id}

 where \textit{node\_id} is the ID of node \textit{i}.

 If more than one \textit{node\_id}s is provided as input, the program
 prints the degree over time of all of them, sequentially.
 
}

\myreference{\refgrowth}
