\myprogram{{tune\_rho}}
          {Construct a multiplex with prescribed inter-layer correlations.} 
          {$<$rank1$>$  $<$rank2$>$ $<$rho$>$ $<$eps$>$ $<$beta$>$ [RND|NAT|INV]}

\mydescription{This programs tunes the inter-layer degree correlation
          coefficient $\rho$ (Spearman's rank correlation) of two
          layers, by adjusting the inter-layer pairing of nodes. The
          files \textit{rank1} and \textit{rank2} are the rankings of
          nodes in the first and second layer, where the n-th line of
          the file contains the rank of the n-th node (the highest
          ranked node has rank equal to 1).

          The parameter \textit{rho} is the desired value of the
          Spearman's rank correlation coefficient, while \textit{eps}
          is the accuracy of \textit{rho}. For instance,
          if \textit{rho} is set equal to -0.25 and \textit{eps} is
          equal to 0.0001, the program stops when the configuration of
          node pairing corresponds to a value of $\rho$ which differs
          from -0.25 by less than 0.0001.

          The parameter \textit{beta} is the typical inverse
          temperature of simulated annealing. 

          If no other parameter is specified, or if the last parameter
          is \texttt{RND}, the program starts from a random pairing of
          nodes. If the last parameter is \texttt{NAT} then the
          program assumes that the initial pairing is the natural one,
          where the nodes have the same ID on both layers. Finally,
          if \texttt{INV} is specified, the initial pairing is the
          inverse pairing, i.e. the one where node 0 on layer 1 is
          paired with node N-1 on layer 2, and so on.

 }


\myreturn{The program prints on \texttt{stdout} a pairing, i.e. a list
of lines in the format:

\hspace{0.5cm} \textit{IDL1 IDL2}

where \textit{IDL1} is the ID of the node on layer 1 and \textit{IDL2}
is the corresponding ID of the same node on layer 2.
}

\myreference{\refcorrelations}
