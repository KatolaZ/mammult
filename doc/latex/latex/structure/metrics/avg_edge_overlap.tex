\myprogram{{avg\_edge\_overlap.py}}
          {compute the average edge overlap of a multiplex.}  
          {$<$layer1$>$ [$<$layer2$>$...]}

\mydescription{Compute and print on output the average edge overlap 

          \begin{equation*} \omega^{*}
          = \frac{\sum_{i}\sum_{j>i}\sum_{\alpha}a_{ij}\lay{\alpha}}{ \sum_{i}\sum_{j>i}(1
          - \delta_{0,\sum_{\alpha}a_{ij}\lay{\alpha}})} \end{equation*}
          
  \noindent i.e., the expected \textit{number} of layers on which an
   edge of the multiplex exists, and the corresponding normalised
   quantity:
  
  \begin{equation*}
      \omega = \frac{\sum_{i}\sum_{j>i}\sum_{\alpha}a_{ij}\lay{\alpha}}{M \sum_{i}\sum_{j>i}(1
      - \delta_{0,\sum_{\alpha}a_{ij}\lay{\alpha}})}
  \end{equation*}
  
  \noindent that is the expected \textit{fraction} of layers on which
  an edge of the multiplex is present.
  
  Each input file contains the (undirected) edge list of a layer, and
  each line is in the format:
  
  \hspace{0.5cm}\textit{src\_ID} \textit{dest\_ID}
  
  where \textit{src\_ID} and \textit{dest\_ID} are the IDs of the two
  endpoints of an edge.}

\myreturn{The program prints on \texttt{stdout} a single line, in the
  format:
  
  \hspace{0.5cm} \textit{omega\_star omega}

  \noindent where \textit{omega\_star} and \textit{omega} are,
  respectively, the expected number and fraction of layers in which an
  edge is present.}

\myreference{\refmetrics

  \vspace{0.5cm}\refvisibility}
