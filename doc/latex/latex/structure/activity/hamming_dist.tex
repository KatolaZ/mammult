\myprogram{{hamming\_dist.py}}
          {compute the normalised Hamming distance between all the pairs of
          layers of a multiplex.}
          {$<$layer1$>$ $<$layer2$>$ [$<$layer3$>$...]}

\mydescription{Compute and print on output the normalised Hamming distance 
 $H_{\alpha, \beta}$ (i.e., the fraction of nodes which are active on
          either of the layers, but not on both) between all pairs of
          layers. The layers are given as input in the
          files \textit{layer1}, \textit{layer2}, etc.
  
  Each input file contains the (undirected) edge list of a layer, and
  each line is in the format:
  
  \hspace{0.5cm}\textit{src\_ID} \textit{dest\_ID}
  
  where \textit{src\_ID} and \textit{dest\_ID} are the IDs of the two
  endpoints of an edge.}

\myreturn{The program prints on \texttt{stdout} a list of lines, in
  the format:

  \hspace{0.5cm} \textit{layer1 layer2 hamm}

  \noindent where \textit{layer1} and \textit{layer2} are the IDs of
  the layers, and \textit{hamm} is the value of the normalised Haming
  distance $H_{layer1, layer2}$. Layers IDs start from zero, are are
  associated to the layers in the same order in which the layer files
  are provided on the command line.}

\myreference{\refcorrelations}
