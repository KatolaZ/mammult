\myprogram{{node\_degree\_vectors.py}}
          {compute the degree vectors of all the nodes of a multiplex network}
          {$<$layer1$>$ [$<$layer2$>$ ...]}

\mydescription{Compute and print on output the degree vectors of all
          the nodes of a multiplex network, whose layers are given as
  input in the files \textit{layer1}, \textit{layer2}, etc.
  
  Each file contains the (undirected) edge list of a layer, and each
  line is in the format:
  
  \hspace{0.5cm}\textit{src\_ID} \textit{dest\_ID}
  
  where \textit{src\_ID} and \textit{dest\_ID} are the IDs of the two
  endpoints of an edge.}

\myreturn{A list of lines, where the n-th line is the 
  vector of degrees of the n-th node, in the format:
  
  \hspace{0.5cm}\textit{noden\_deg\_lay1 noden\_deg\_lay2 ... noden\_deg\_layM}

  \noindent As usual, node IDs start from zero and proceed
  sequentially, without gaps, i.e., if a node ID is not present in any
  of the layer files given as input, the program considers it as being
  isolated on all the layers.

}

\myreference{\refgrowth\\ \\ \indent \refmetrics}

